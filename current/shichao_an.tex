% resume.tex
% vim:set ft=tex spell:

\documentclass[13pt,letterpaper]{article}
\usepackage[letterpaper,top=0.4in,left=0.4in,right=0.4in,bottom=0.4in]{geometry}
\usepackage{tabularx}
%\usepackage[scaled=0.92]{helvet}

\usepackage[utf8]{inputenc}
\usepackage{mdwlist}
\usepackage[LGR,OT1]{fontenc}

% Font control
%\usepackage{epigrafica}
% Open Sans
%\usepackage[default,osfigures,scale=0.95]{opensans} 
\usepackage[default,scale=0.95]{opensans}
\usepackage[T1]{fontenc}
\usepackage{textcomp}

\usepackage[usenames,dvipsnames]{xcolor}
\usepackage{hyperref}
\usepackage{hyphenat}

% For multi-line commen
\usepackage{verbatim}
%\usepackage{tgpagella}
\pagestyle{empty}
\definecolor{dgray}{RGB}{80,80,80}
\setlength{\tabcolsep}{0em}

% indentsection style, used for sections that aren't already in lists
% that need indentation to the level of all text in the document
\newenvironment{indentsection}[1]%
{\begin{list}{}%
    {\setlength{\leftmargin}{#1}}%
    \item[]%
}
{\end{list}}

% opposite of above; bump a section back toward the left margin
\newenvironment{unindentsection}[1]%
{\begin{list}{}%
    {\setlength{\leftmargin}{-0.5#1}}%
    \item[]%
}
{\end{list}}

% format two pieces of text, one left aligned and one right aligned
\newcommand{\headerrow}[2]
{\begin{tabular*}{\linewidth}{l@{\extracolsep{\fill}}r@{\hspace{0.6em}}}
    #1 &
    #2 \\
\end{tabular*}}

\newcommand{\slist}[1]
{
\vspace{-1.8em}
\subsection*{\color{dgray} #1}
\vspace{-0.4em}
}

\newcommand{\sbullet}[1] { \item[-] \leftskip-1em \rightskip2.8cm #1}

\newcommand{\linktitle}[2]
{ \textbf{\href{#1}{\color{NavyBlue}#2}}}

\newcommand{\link}[2]
{
	\textbf{\href{#1}{#2}}
}
\newcommand{\bigtitle}[1]
{
	\vspace{-1.1em} \\
	#1
	\vspace{-1.2em} \\
}
\newcommand{\subtitle}[1]
{
	\vspace{-1.1em} \\
	{\small{#1}}
}

\newcommand{\shortdesc}[1]
{\small \color{dgray}, #1}

\newenvironment{narrowitems}
{\begin{itemize*} \vspace{-0.4em}}
{\vspace{-0.2em} \end{itemize*}}

% make "C++" look pretty when used in text by touching up the plus signs
%\newcommand{\CPP}
%{C\nolinebreak[4]\hspace{-.05em}\raisebox{.22ex}{\small\bf ++}}

% and the actual content starts here
\begin{document}
\hyphenation{library features}
% Control font weight
% \fontseries{l}

\begin{center}
{\huge \textbf{Shichao An}}
\\
\small \vspace{0.4em} \color{dgray}
%30 Newport Parkway\ \ \textbullet
%\ \ Apt 3214\ \ \textbullet
%\ \ Jersey City, NJ 07310
(631) 464-2914\ \ \textbullet
\ \ \href{mailto:shichao.an@nyu.edu}{shichao.an@nyu.edu}\\
\ \ \href{http://www.shichao-an.info/}{\color{NavyBlue} www.shichao-an.info} \textbullet
\ \ \href{https://github.com/shichao-an}{\color{NavyBlue} github.com/shichao-an}
\end{center}
%\hrule

\slist{EDUCATION}
\begin{itemize}
	\parskip=0.1em
	
    \item
    \headerrow
	{\bigtitle{\textbf{New York University}}}
    {\emph{\color{dgray} \small 09/2012 -- 05/2014 \color{dgray}(exp.)}}
	\subtitle{M.S. Candidate in Computer Science\shortdesc{GPA: 3.8/4.0}}

    \begin{narrowitems}
		 \sbullet \textbf{Relevant Courses}: Programming Languages, Fundamental Algorithms, Operating Systems, Database Systems, Cloud Computing, Open Source Tools, Multicore Processors
    \end{narrowitems}

    \item
    \headerrow
	{\textbf{University of Science and Technology Beijing} \color{dgray} \small (Beijing, China)} 
	{\emph{\color{dgray} \small 09/2008 -- 06/2012}}
	\subtitle{B.S. in Computer Science\shortdesc{GPA: 3.6/4.0}}

\end{itemize}

\slist{PROJECTS}

\begin{itemize}
    \parskip=0.1em

    \item
    \headerrow
	{\bigtitle{\linktitle{https://github.com/shichao-an/adium-sh}{Adium Shell}\shortdesc{Shell Utilities and Python Wrapper for Adium IM}}}
    {\emph{\color{dgray} \small 03/2014 -- present}}
    \begin{narrowitems}
		\sbullet Implemented shell utilities, wrapper interface and AppleScript scripts that access Adium features
		\sbullet Extended the interface with event handling by parsing chat logs, using Watchdog library that leveraged Kqueue and FSEvents on OS X
		\sbullet Enabled automatic conversation feature that supports pattern-based and user-extended chat methods, and external chatterbot APIs such as SimiSimi 
    \end{narrowitems}

    \item
    \headerrow
	{\bigtitle{\linktitle{http://geoport.co}{GeoPort}\shortdesc{Geo-Social Platform}}}
    {\emph{\color{dgray} \small 11/2013 -- present}}
    \begin{narrowitems}
		\sbullet Designed architecture and implemented applications with Django, MongoDB and MongoEngine
		\sbullet Facilitated on-map real-time location sharing with Google Maps and client-side JavaScript
        \sbullet Conducted software configuration management and maintained documentation wiki for the team
    \end{narrowitems}
    
    \item
    \headerrow
	{\bigtitle{\linktitle{https://github.com/shichao-an/soundmeter}{SoundMeter}\shortdesc{Simple Command-Line Sound Level Meter}}}
    {\emph{\color{dgray} \small 12/2013 -- 01/2014}}   
    \begin{narrowitems}
		\sbullet Built a command-line tool able to obtain sound level in real time with audio input devices based on PortAudio and PyAudio libraries
		\sbullet Implemented the meter that supports event triggering and the extensible monitor API
    \end{narrowitems}


    \item
    \headerrow
	{\bigtitle{\linktitle{https://github.com/shichao-an/PPP}{Python and Parallel Programming}\shortdesc{Multicore Processors Course Project}}}
    {\emph{\color{dgray} \small 10/2013 -- 12/2013}}
    \begin{narrowitems}
		\sbullet Experimented writing and profiling parallelized programs with \emph{threading}, \emph{multiprocessing} modules of the standard library and OpenMP in Cython
		\sbullet Analyzed different types of parallelism and compared the approaches in terms of performance and development efficiency 
    \end{narrowitems}

\end{itemize}


\slist{EXPERIENCE}

\begin{itemize}
    \parskip=0.1em

    \item
    \headerrow
	{\linktitle{http://signl.com/}{SIGNL}   \color{dgray} \small (New York)}
    {\emph{\color{dgray} \small 06/2013 -- 08/2013}}
    \subtitle{Software Development Intern}

    \begin{narrowitems}
        \sbullet Developed site features and reusable applications related to authentication, content management, request security, and feed system using Django
		\sbullet Used Redis cache to implement the activity stream system that records company page views and follows, pushing notifications to users
		\sbullet Implemented news processing component able to highlight keyword sentences for the feed page
		\sbullet Wrote an adaptive web crawler that scraps social accounts and mobile app identifiers
    \end{narrowitems}

    \item
    \headerrow
    {\linktitle{http://www.ibeike.com/}{iBeiKe.com}  \color{dgray} \small (Beijing, China)}
    {\emph{\color{dgray} \small 07/2011 -- 06/2012}}
    \subtitle{System Administrator \color{dgray} (Department of Network)}

    \begin{narrowitems}
        \sbullet Deployed upgrades, extensions and maintenance scripts for content management systems (PHP) hosted on the Linux server
        \sbullet Automated the formatting and uploads of mass text and captured news from official sources to the MediaWiki-based subsite, iBeiKe Wiki, through API with Python and Pywikibot
    \end{narrowitems}

\end{itemize}


\slist{SKILLS}

\begin{indentsection}{\parindent}

\begin{tabularx} 
	{\textwidth}{l @{\hspace{2em}}l l}
	\textbf{\color{dgray}Programming Languages:} & & \textbf{\color{dgray}Web Development:} HTML5, Django, jQuery, Bootstrap \\[0.3ex]
	\hspace{1em}{\color{dgray} Advanced:} Python, C, Bash scripting & & \textbf{\color{dgray}Databases:} MySQL, MongoDB, Oracle\\[0.3ex]
	\hspace{1em}{\color{dgray} Intermediate:} JavaScript, Java, PHP, Perl  & \hspace{1em} &\textbf{\color{dgray}Operating Systems:}  Linux (Fedora/RHEL/Ubuntu), OS X\\[0.3ex]
\end{tabularx}
\textbf{\color{dgray}Software and Technologies:} Vim, Git, KVM, Puppet, Selenium, Amazon Web Services
\end{indentsection}

\slist{CERTIFICATIONS}

\begin{itemize}
    \parskip=-0.1em
	\item{\linktitle{https://www.redhat.com/wapps/training/certification/verify.html?certNumber=140-025-057\&isSearch=False\&verify=Verify}{Red Hat Certified Engineer} (RHCE)}
	\item{\linktitle{https://www.redhat.com/wapps/training/certification/verify.html?certNumber=140-025-057\&isSearch=False\&verify=Verify}{Red Hat Certified System Administrator} (RHCSA)}

\end{itemize}
\end{document}
