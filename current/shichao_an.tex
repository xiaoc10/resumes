% resume.tex
% vim:set ft=tex spell:

\documentclass[13pt,letterpaper]{article}
\usepackage[letterpaper,top=0.6in,left=0.9in,right=0.9in,bottom=0.6in]{geometry}
\usepackage{tabularx}
%\usepackage[scaled=0.92]{helvet}

\usepackage[utf8]{inputenc}
\usepackage{mdwlist}
%\usepackage[T1]{fontenc}
\usepackage[LGR,OT1]{fontenc}

% Font control
\usepackage{epigrafica}
% Open Sans
%\usepackage[default,osfigures,scale=0.95]{opensans} 

\usepackage{textcomp}

\usepackage[usenames,dvipsnames]{xcolor}
\usepackage{hyperref}

% For multi-line commen
\usepackage{verbatim}
%\usepackage{tgpagella}
\pagestyle{empty}
\definecolor{dgray}{RGB}{80,80,80}
\setlength{\tabcolsep}{0em}

% indentsection style, used for sections that aren't already in lists
% that need indentation to the level of all text in the document
\newenvironment{indentsection}[1]%
{\begin{list}{}%
    {\setlength{\leftmargin}{#1}}%
    \item[]%
}
{\end{list}}

% opposite of above; bump a section back toward the left margin
\newenvironment{unindentsection}[1]%
{\begin{list}{}%
    {\setlength{\leftmargin}{-0.5#1}}%
    \item[]%
}
{\end{list}}

% format two pieces of text, one left aligned and one right aligned
\newcommand{\headerrow}[2]
{\begin{tabular*}{\linewidth}{l@{\extracolsep{\fill}}r}
    #1 &
    #2 \\
\end{tabular*}}

% make "C++" look pretty when used in text by touching up the plus signs
\newcommand{\CPP}
{C\nolinebreak[4]\hspace{-.05em}\raisebox{.22ex}{\small\bf ++}}

% and the actual content starts here
\begin{document}

\begin{center}
{\huge \normalfont \textbf{Shichao An}}
\small \\\vspace{0.4em} \color{dgray}
30 Newport Parkway\ \ \textbullet
\ \ Apt 3214\ \ \textbullet
\ \ Jersey City, NJ 07310
\\
(631) 464-2914\ \ \textbullet
\ \ \href{mailto:shichao.an@nyu.edu}{shichao.an@nyu.edu}\\
\ \ \href{http://www.shichao-an.info/}{\color{NavyBlue} www.shichao-an.info}
\end{center}

%\hrule
\vspace{-1.0em}
\subsection*{\color{NavyBlue} EDUCATION}

\begin{itemize}
    \parskip=0.1em

    \item
    \headerrow
        {\textbf{New York University}}
        {\emph{\color{dgray} \small 09/2012 -- 05/2014 \color{dgray}(exp.)}}
    \headerrow
        {\color{dgray}  \small Courant Institute of Mathematical Sciences\\ 
         \small M.S. Candidate in Computer Science}
         
    \begin{itemize*} \small
         \item[-]\leftskip-1em \rightskip2.8cm \small \textbf{Relevant Courses}: Programming Languages, Fundamental Algorithms, Operating Systems, Database Systems, Cloud Computing, Open Source Tools, Multicore Processors
         \item[-]\leftskip-1em \rightskip2.8cm \small \textbf{Current GPA}: 3.8/4.0
         
    \end{itemize*}
         
    \item
    \headerrow
        {\textbf{University of Science and Technology Beijing} \color{dgray} \small (Beijing, China)} 
        {\emph{\color{dgray} \small 09/2008 -- 06/2012}}
    \headerrow
        {\color{dgray} \small School of Computer and Communication Engineering\\ 
         \small B.S. in Computer Science}

    \begin{itemize*} \small
         \item[-]\leftskip-1em \rightskip2.8cm \small \textbf{Major GPA}: 3.6/4.0
         
    \end{itemize*}

\end{itemize}

%\hrule
\vspace{-1.4em}
\subsection*{\color{NavyBlue} PROJECTS}

\begin{itemize}
    \parskip=0.1em
    
    \item
    \headerrow
        {\textbf{\href{https://github.com/shichao-an/GeoPort}{\color{NavyBlue}GeoPort}} (\href{http://geoport.co/}{\color{NavyBlue} geoport.co})}
        {\emph{\color{dgray} \small 11/2013 -- present}}
        
    \begin{itemize*} \small
        \item[-]\leftskip-1em \rightskip2.8cm Built a geo-social platform that facilitates event-based real-time location sharing
        \item[-]\leftskip-1em \rightskip2.8cm Utilized APIs of Google Maps and Google Geocoding and implemented the website with Django, MongoDB and Bootstrap
        
    \end{itemize*}
    
    \item
    \headerrow
        {\textbf{\href{https://github.com/shichao-an/soundmeter}{\color{NavyBlue}SoundMeter}}}
        {\emph{\color{dgray} \small 12/2013 -- 01/2014}}   
        
    \begin{itemize*} \small
        \item[-]\leftskip-1em \rightskip2.8cm Implemented a command-line tool to obtain sound power in real time, which basically turns the audio recording functionality into a sound meter for machines that ship with audio input devices
    \end{itemize*}


    \item
    \headerrow
        {\textbf{\href{https://github.com/shichao-an/PPP}{\color{NavyBlue}{{Python and Parallel Programming}}}}}
        {\emph{\color{dgray} \small 10/2013 -- 12/2013}}

    \begin{itemize*} \small 
        \item[-]\leftskip-1em \rightskip2.8cm Compared and analyzed parallelism in CPU-bound, I/O-bound and memory-bound applications implemented with \verb#threading#, \verb#multiprocessing# of the standard library, and OpenMP/Cython
    \end{itemize*}


\end{itemize}

%\hrule
\vspace{-1.4em}
\subsection*{\color{NavyBlue} EXPERIENCE}

\begin{itemize}
    \parskip=0.1em

    \item
    \headerrow
        {\textbf{\href{http://signl.com/}{\color{NavyBlue}SIGNL}}   \color{dgray} \small (New York)}
        {\emph{\color{dgray} \small 06/2013 -- 08/2013}}

    \headerrow
        { \small Software Development Intern }

    \begin{itemize*} \small
        \item[-]\leftskip-1em \rightskip2.8cm Developed site backends, reusable applications and front-end features, including authentication, content management, activity stream, news processing, security and SEO
        \item[-]\leftskip-1em \rightskip2.8cm Used Django, MySQL, MongoDB, Redis, Celery and Node.js for the backend implementation of major site functionalities and real-time data aggregation
            \end{itemize*}

    \item
    \headerrow
        {\textbf{\href{http://www.ibeike.com/}{\color{NavyBlue}iBeiKe.com}}  \color{dgray} \small (Beijing, China)}
        {\emph{\color{dgray} \small 07/2010 -- 06/2012}}

    \headerrow
        { \small System Administrator \color{dgray} (Department of Network)}

    \begin{itemize*} \small
        \item[-]\leftskip-1em \rightskip2.8cm Deployed upgrades, extensions and maintenance scripts for Web applications
        \item[-]\leftskip-1em \rightskip2.8cm Automated mass text uploads and news capture from official sources to the MediaWiki-based subsite, iBeiKe Wiki, with Python and Pywikibot 
    \end{itemize*}

\end{itemize}

%\hrule
\vspace{-1.4em}
\subsection*{\color{NavyBlue} SKILLS}

\begin{indentsection}{\parindent}

\small
\begin{tabularx} 
{\textwidth}{l l l}
 \textbf{Programming Languages:} & & \textbf{Web Development:} HTML5, Django, Node.js, jQuery \\
 \hspace{1em}{\color{dgray} Proficient:} Python, C, JavaScript, shell scripting & & \textbf{Databases:} MySQL, Oracle, MongoDB\\
 \hspace{1em}{\color{dgray} Intermediate:} Java, C\verb$++$, PHP, Ruby, Perl  & \hspace{1em} &\textbf{Operating Systems:}  Linux (Ubuntu/RHEL), OS X
\end{tabularx}
\textbf{Software and Tools:} Git, \LaTeX, Vim, KVM, Amazon Web Services, WordPress\\


\end{indentsection}
\end{document}
